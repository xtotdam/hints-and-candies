\documentclass[a4paper,12pt]{report}
\usepackage[T2A]{fontenc}
\usepackage[utf8]{inputenc}
\usepackage{amssymb,amsfonts}
\usepackage[fleqn]{amsmath}
\usepackage[russian,english]{babel}
\usepackage{anyfontsize}
\usepackage{graphicx}
\usepackage{fancyhdr}
\usepackage{cmap}
% \usepackage{times}
\usepackage{mathrsfs}
\usepackage[inline]{enumitem}
\usepackage[usenames,dvipsnames,svgnames,table]{xcolor}
\setlength{\headheight}{15.2pt}
\renewcommand{\headrulewidth}{0pt}
\pagestyle{fancy}

\usepackage{geometry}
\geometry{left=2cm}
\geometry{right=1.5cm}
\geometry{top=1cm}
\geometry{bottom=2cm}

\let\divsymb=\div % rename builtin command \div to \divsymb
\let\underdot=\d % rename builtin command \d{} to \underdot{}
\let\vaccent=\v % rename builtin command \v{} to \vaccent{}

\DeclareMathOperator{\sinc}{sinc}
\DeclareMathOperator{\const}{const}
\DeclareMathOperator{\Tr}{Tr}   %trace
\DeclareMathOperator{\Dim}{dim} %dimensions

\renewcommand{\v}[1]{\ensuremath{\mathbf{#1}}} % for vectors
\newcommand{\gv}[1]{\ensuremath{\mbox{\boldmath$ #1 $}}} % for vectors of Greek letters
\newcommand{\uv}[1]{\ensuremath{\mathbf{\hat{#1}}}} % for unit vector

\newcommand{\abs}[1]{\left| #1 \right|} % for absolute value
\newcommand{\norm}[1]{\lVert #1 \rVert} % for norm ||f||
\newcommand{\qnorm}[1]{\lVert #1 \rVert ^2} % for norm quadrat ||f||^2
\newcommand{\avg}[1]{\left< #1 \right>} % for average

\newcommand{\vmult}[2]{[\vec{#1}{\times}\vec{#2}]} % vector multiplication
\newcommand{\scmult}[2]{(\vec{#1}{\cdot}\vec{#2})} % scalar multiplication

\renewcommand{\d}[2]{\frac{d #1}{d #2}} % for derivatives
\newcommand{\dd}[2]{\frac{d^2 #1}{d #2^2}} % for double derivatives
\newcommand{\pd}[2]{\frac{\partial #1}{\partial #2}} % for partial derivatives
\newcommand{\pdd}[2]{\frac{\partial^2 #1}{\partial #2^2}} % for double partial derivatives
\newcommand{\pddv}[3]{\frac{\partial^2 #1}{\partial #2 \partial #3}} % for double partial derivatives, diff. variables
\newcommand{\pdc}[3]{\left( \frac{\partial #1}{\partial #2} \right)_{#3}} % for thermodynamic partial derivatives

\newcommand{\ket}[1]{\left| #1 \right>} % for Dirac kets
\newcommand{\bra}[1]{\left< #1 \right|} % for Dirac bras
\newcommand{\braket}[2]{\left< #1 \vphantom{#2} \right|\left. \!#2 \vphantom{#1} \right>} % for Dirac brackets
\newcommand{\matrixel}[3]{\left< #1 \vphantom{#2#3} \right| #2 \left| #3 \vphantom{#1#2} \right>} % for Dirac matrix elements

\newcommand{\comml}[2]{\left[ \hat{#1}, #2 \right]} % [A,...]
\newcommand{\commr}[2]{\left[ #1, \hat{#2} \right]} % [...,B]
\newcommand{\comm}[2]{\left[ \hat{#1}, \hat{#2} \right]} % for commutator for only two operators like [A,B]
\newcommand{\commi}[4]{\left[ \hat{#1}_{#2}, \hat{#3}_{#4} \right]} % for commutator for only two operators like [A,B] with indices

\renewcommand{\div}[1]{\operatorname{div}\vec{#1}} % dirty hack for divergence
\newcommand{\rot}[1]{\operatorname{rot}\vec{#1}}
\newcommand{\grad}[1]{\operatorname{grad}\vec{#1}}

\newcommand{\gradn}[1]{\vec{\nabla} #1} % for gradient
\newcommand{\divn}[1]{\scalmult{\nabla}{#1}} % for divergence
\newcommand{\rotn}[1]{\vecmult{\nabla}{#1}} % for curl

\newcommand{\praiseme}[1]{\scalebox{.4}{ \textcircled{\scalebox{.5}{CC}} \textcircled{\scalebox{.5}{BY}} \textcircled{\scalebox{.5}{SA}} #1, \the\day.\the\month.\the\year}}
\newcommand{\tobewritten}[0]{\textcolor{red}{\textsc{To be written}}}
\newcommand{\NB}[0]{\textbf{NB!}}
\newcommand{\Def}[0]{$\mathfrak{Def.}$}
\newcommand{\Th}[0]{$\mathfrak{[Th.]}$}
\newcommand{\Wiki}[0]{$\mathfrak{(Wiki)}$}

\newcommand{\bbar}[1]{\bar{\bar{#1}}}

\makeatletter
\newcommand{\inlineitem}[1][]{
    \ifnum\enit@type=\tw@
        {\descriptionlabel{#1}}
        \hspace{\labelsep}
    \else
        \ifnum\enit@type=\z@
        \refstepcounter{\@listctr}\fi
    \quad\@itemlabel\hspace{\labelsep}%
    \fi
}
\makeatother

\begin{document}
\rfoot{\praiseme{Xtotdam}}

\fontsize{20}{24}
\center{\bfseries Квантовая теория - теоретический минимум}
\line(1,0){500}
\fontsize{12}{15}
\begin{enumerate}[label=\textbf{\underline{\arabic*.}}]
\item Матрица плотности  \begin{itemize}
            \item условие нормировки для матрицы плотности $\hat{\rho}$:
            $ \text{Tr}\hat{\rho} = 1 $
            \item среднее значение наблюдаемой $\avg{\hat{A}}$, если система находится в состоянии с матрицей плотности $\hat{\rho}$:
            $ \avg{\hat{A}}_\rho = \text{Tr}\hat{A}\hat{\rho} $
            \item вероятность пребывания в чистом состоянии $\ket{\psi}$, если система находится в состоянии с матрицей плотности $\hat{\rho}$:
            $ P_{\ket{\psi}} = \matrixel{\psi}{\hat{\rho}}{\psi} $
            \item необходимое и достаточное условие чистоты состояния, если система находится в состоянии с матрицей плотности $\hat{\rho}$:
            $ \hat{\rho}^2 = \hat{\rho}$ \\
            cвязь между $\hat{\rho}$ и волновой функцией $\ket{\psi}$ в этом случае:
            $ \hat{\rho} = \ket{\psi}\bra{\psi} $
        \end{itemize}
\item Волновая функция  \begin{itemize}
            \item условие нормировки волновой функции $\ket{\psi}$:
            $ \braket{\psi}{\psi}=1 $
            \item среднее значение наблюдаемой $\avg{\hat{A}}$, если система находится в состоянии с волновой функцией $\ket{\psi}$:
            $ \avg{\hat{A}}_{\ket{\psi}} = \matrixel{\psi}{\hat{A}}{\psi} $
            \item вероятность пребывания в чистом состоянии $\ket{\xi}$, если система находится в состоянии с волновой функцией $\ket{\psi}$:
            $ P_{\ket{\xi}} = \abs{\braket{\xi}{\psi}}^2 $
        \end{itemize}
\item Измерение наблюдаемой $\hat{A}$ (чисто дискретный спектр) в состоянии $\hat{\rho}$  \begin{itemize}
            \item вероятность получить значение $a_i$, если система находится в состоянии с матрицей плотности $\hat{\rho}$: \\
            $ \hat{A}\ket{\alpha_n}=a_n\ket{\alpha_n};\; P_{a_n} = \matrixel{\alpha_n}{\hat{\rho}}{\alpha_n} $
            \item вероятность получить значение $a_i$, если система находится в состоянии с волновой функцией $\ket{\psi}$: \\
            $ \text{det}(\hat{A}-a_n E)\ket{a} = 0 \Rightarrow \ket{\xi_{a_n}};\; P_{a_n} = \abs{\braket{\xi_{a_n}}{\psi}}^2 $
        \end{itemize}
\item Составные системы  \begin{itemize}
            \item выражение для матрицы плотности подсистемы:
            $ \rho_1 = \text{Tr}_2 \rho;\; \rho_2 = \text{Tr}_1 \rho $
        \end{itemize}
\item Динамика  \begin{itemize}
            \item Уравнение Гайзенберга для произвольного оператора $\hat{A}$:
            $ \d{\hat{A}(t)}{t}=\frac{i}{\hbar}\comm{H}{A(t)} $
            \item Нестационарное уравнение Шредингера (общий случай):
            $ i\hbar \d{}{t}\ket{\psi(t)} = \hat{H}\ket{\psi(t)} $
            \item Стационарное уравнение Шредингера (общий случай):
            $ \hat{H}\ket{\psi} = E\ket{\psi} $
        \end{itemize}
\item Одномерное движение материальной точки  \begin{itemize}
            \item каноническое коммутационное соотношение:
            $ \comm{x}{p} = i\hbar $
            \item нестационарное уравнение Шредингера в координатном представлении: \\
            $ i\hbar \pd{}{t}\psi(\v{r},t) = -\frac{\hbar^2}{2m}\triangle\psi(\v{r},t)+V(\v{r},t)\psi(\v{r},t) $
            \item стационарное уравнение Шредингера в координатном представлении: \\
            $ -\frac{\hbar^2}{2m}\triangle\psi(\v{r}) + (V(\v{r})-E)\psi(\v{r}) = 0 $
            \item уравнение непрерывности:
            $ \text{div}\ \v{j} + \pd{\rho}{t} = 0 $
        \end{itemize}
\item Гармонический осциллятор  \begin{itemize}
            \item
            $ \comm{a}{a^\dagger} = 1 $
            \item
            $ \hat{a}\ket{n} = \sqrt{n}\ket{n-1};\; \hat{a}^\dagger\ket{n} = \sqrt{n+1}\ket{n+1} $
            \item уровни энергии:
            $ E_n = \hbar\omega(n+\frac12) $
            \item когерентное состояние $\ket{\alpha}$: \\
            $ \hat{a}\ket{\alpha} = \alpha\ket{\alpha};\; \bra{\alpha}\hat{a}^\dagger = \bra{\alpha}\alpha^* $
        \end{itemize}
\item Трехмерное движение материальной точки  \begin{itemize}
            \item канонические коммутационные соотношения:
            $ \commi{x}{i}{p}{j} = i\hbar\delta_{ij} $
            \item нестационарное уравнение Шредингера в координатном представлении: \\
            $ i\hbar \pd{}{t}\psi(x,t) = -\frac{\hbar^2}{2m}\triangle\psi(x,t)+V(x,t)\psi(x,t) $
            \item уравнение непрерывности:
            $ \pd{j}{x} + \pd{\rho}{t} = 0 $
        \end{itemize}
\item Момент  \begin{itemize}
            \item определение момента:
            $ \commi{l}{i}{l}{j} = i\epsilon_{ijk}\hat{l}_k $
            \item
            $ \braket{l'm'}{lm} = \delta_{ll'}\delta_{mm'};\; \hat{\v{l}}^2\ket{lm}=l(l+1)\ket{lm};\; \hat{l}_z\ket{lm}=m\ket{lm} $\\
            $ l_{+}\ket{lm}=\ket{l,m+1}\sqrt{(l-m)(l+m+1)} $\\
            $ l_{-}\ket{lm}=\ket{l,m-1}\sqrt{(l+m)(l-m+1)}$
            \item определение скалярного и векторного операторов: \\
            $ \commi{l}{i}{A}{} = 0 \Rightarrow \hat{A} $ -- скаляр \\
            $ \commi{l}{i}{B}{j} = i\epsilon_{ijk}\hat{B}_k \Rightarrow \hat{B} $ -- вектор
            \item матричные элементы скалярного оператора $\hat{A}$:
            $ \matrixel{l'm'}{\hat{A}}{lm} = \delta_{ll'}\delta_{mm'}a(l) $
        \end{itemize}
\item Формулы для операторов  \begin{itemize}
            \item
            $ e^{z\hat{A}}\hat{B}e^{-z\hat{A}} = \hat{B} + \frac{z}{1!}\underbrace{\comm{A}{B}}_{K_1} + \frac{z^2}{2!}\underbrace{\comm{A}{K_1}}_{K_2} = \hat{B}+\sum\limits_{n=1}^{\infty}\frac{z^n}{n!}\comm{A}{K_n} $
            \item
            $ \comm{A}{B}=\lambda \Rightarrow \comml{A}{\hat{f}(\hat{B})}=\lambda \pd{\hat{f}(\hat{B})}{\hat{B}} $
            \item явный вид матриц Паули:
            $ \hat{\sigma}_1=\begin{pmatrix} 0&1\\1&0 \end{pmatrix};\; \hat{\sigma}_2=\begin{pmatrix} 0&-i\\i&0 \end{pmatrix};\; \hat{\sigma}_3=\begin{pmatrix} 1&0\\0&-1 \end{pmatrix} $
            \item
            $ (\v{a}\hat{\vec{\sigma}})(\v{b}\hat{\vec{\sigma}}) = \hat{E}(\v{a},\v{b}) + i([\v{a}\times\v{b}],\hat{\vec{\sigma}}) $
        \end{itemize}
\end{enumerate}

\begin{enumerate}[label=\textbf{\underline{\arabic*.}}]
\item Стационарная теория возмущений  \begin{itemize}
            \item Условие применимости: \\
            $ \abs{V_{ml}} \ll \abs{E_m^0 - E_l^0}\, \forall m,l;\; V_{ml}=\matrixel{\psi_m^{(0)}}{\hat{V}}{\psi_l^{(0)}} $
            \item Невырожденный уровень. Поправка к энергии, 1-й и 2-й порядки: \\
            $ E_l^{(1)} = V_{ll};\; E_l^{(2)} = \sum_{n \neq l} \frac{\abs{V_{ln}}^2}{E_l^{(0)} - E_n^{(0)}} $
            \item Невырожденный уровень. Поправка к волновой функции, 1-й порядок: \\
            $ \ket{\psi_n^{(1)}} = \sum_{m \neq n} \frac{\matrixel{\psi_m^{(0)}}{\hat{V}}{\psi_n^{(0)}}}{E_n^{(0)} - E_m^{(0)}} $
            \item Вырожденный уровень. Поправка к энергии, 1-й порядок: \\
            $ \text{det}(V_{\alpha\beta} - \delta_{\alpha\beta}E^{(1)})=0 \Rightarrow E^{(1)} $ -- секулярное уравнение

        \end{itemize}
\item Потенциальное рассеяние  \begin{itemize}
            \item Амплитуда рассеяния в 1-м Борновском приближении: \\
            $ f(\v{q}) = -\frac{m}{2\pi\hbar^2}\int e^{-i\v{q}\v{x}} V(\v{x})d\v{x} $ \\
            Условия применимости 1-го Борновского приближения: \\
            $ md^2\bar{V} \ll \hbar^2;\; \bar{V} = \frac1{2\pi d^2} \abs{\int \frac1{\v{r}}V(\v{r})d\v{r}};\; kd \ll 1 $ -- малые энергии \\
            $ md^2\tilde{V} \ll \hbar^2;\; \tilde{V} = \frac1{d} \abs{\int V(\v{r})d\v{r}};\; kd \gg 1 $ -- высокие энергии \\
            \item Условие унитарности для парциальных амплитуд рассеяния: \\
            $ \Im{f_l} = k \abs{f_l}^2 $
            \item Выражение для парциальной амплитуды рассеяния через фазу рассеяния: \\
            $ f_l = \frac1{2ik}(2l+1)(e^{2i\delta_l} - 1) $
            \item Асимптотика для решения радиального уравнения Шредингера в задаче рассеяния: \\
            $ R_l(r \rightarrow \infty) = i^l e^{i\delta_l} \frac1{kr} \cos(kr+\delta_l - \frac{\pi(l+1)}{2}) $

        \end{itemize}
\item Переходы  \begin{itemize}
            \item Переходы мгновенные и адиабатические: определение и результат: \\
            Мгновенные переходы: $ \hat{H} = \hat{H}_0 $ при $ t=0 $, потом скачок $ \hat{H} = \hat{H}_0 + \hat{H}_I\,:\, \tau \ll \frac1{\omega_{fi}} $ \\
            $ P_{f'} = \abs{\frac{\matrixel{\psi_f}{\hat{H}_I}{\psi_i}}{E_f-E_i}}^2 $ \\
            Адиабатические переходы: то же самое, но переход плавный без скачка\,:\, $ \tau \gg \frac1{\omega_{fi}} $\\
            ``Адиабатические переходы --- это те, которые не происходят.'' Силаев
            \item Вероятность перехода в первом порядке нестационарной теории возмущений: \\
            $ P_f(t) = \abs{C_f(t)}^2;\; C_f^{(0)}=\delta_{fi};\; C_f^{(1)}=-\frac{i}{\hbar}\int_0^t e^{i\omega_{fi}t}\matrixel{\psi_f}{\hat{H}_I(t)}{\psi_i}dt;\;\omega_{ii}=0 $
            \item Уравнение эволюции волновой функции в представлении взаимодействия (Дирака): \\
            $ \psi=e^{iH_0t}\varphi;\; i\d{\psi}{t}=(\hat{H}_0+\hat{H}_I)\psi \Rightarrow i\d{\varphi}{t}=e^{iH_0t}\hat{H}_I e^{iH_0t}\varphi $
            \item Золотое правило Ферми: \\
            $ P_{fi} = \frac{2\pi}{\hbar}\abs{\matrixel{\psi_f}{\hat{V}}{\psi_i}}^2\rho(E_f) $ -- постоянное возмущение\\
            $ P_{fi} = \frac{2\pi}{\hbar}\abs{\matrixel{\psi_f}{\hat{\Omega}_\pm}{\psi_i}}^2\delta(E_f-E_i\pm \hbar\omega) $ -- периодическое возмущение $ \hat{V}_\pm=\hat{\Omega}_\pm e^{\pm i\omega t} $

        \end{itemize}
\item Вторичное квантование.  \begin{itemize}
            \item Канонические коммутационные соотношения для операторов рождения и уничтожения: \\
            $ \commi{a}{m}{a^\dagger}{n} = \delta_{mn};\; \commi{a}{m}{a}{n} = \commi{a^\dagger}{m}{a^\dagger}{n} = 0 $
            \item Оператор волновой функции: \\
            $ \hat{\psi}(x) = \sum_n \psi_n(x)\hat{a}_n;\; \psi_n = \braket{n}{\psi} $
            \item Выражения для одночастичного и двухчастичного операторов: \\
            $ \hat{V} = \int dx\ \hat{\psi}^\dagger(x)V(x)\hat{\psi}(x) = \sum_{kl} V_{kl}\ \hat{a}^\dagger_k\ \hat{a}_l $ -- одночастичный \\
            $ \hat{W} = \frac12 \int dxdy\ \hat{\psi}^\dagger(x)\hat{\psi}^\dagger(y)W(x,y)\hat{\psi}(y)\hat{\psi}(x) = \sum_{klmn} \frac12 W_{klmn}\ \hat{a}^\dagger_k\ \hat{a}^\dagger_l\ \hat{a}_m\ \hat{a}_n $ -- двухчастичный

        \end{itemize}
\item Излучение  \begin{itemize}
            \item Коммутационные соотношения для операторов рождения и уничтожения фотонов: \\
            $ \commi{a}{\v{k}p}{a^\dagger}{\v{k}'p'} = \delta_{\v{k}\v{k}'}\delta_{pp'};\; \commi{a}{\v{k}p}{a}{\v{k}'p'} = 0 $
            \item Энергия и импульс поля излучения: \\
            $ \hat{H} = \sum_{\v{k},p}\hbar\omega_\v{k}\ \hat{a}^\dagger_{\v{k},p}\ \hat{a}_{\v{k},p} $\\
            $ \hat{\v{p}} = \sum_{\v{k},p}\hbar\v{k}\ \hat{a}^\dagger_{\v{k},p}\ \hat{a}_{\v{k},p} $\\
            \item Оператор вектор-потенциала: \\
            $ \hat{\v{A}}_\Gamma(\v{x},t) = \sum_{\v{k},p}\sqrt{\frac{2\pi\hbar c^2}{\omega_\v{k}L^3}}\cdot e_{\v{k},p}(\hat{a}_{\v{k},p}\cdot e^{i\v{k}\v{x}-i\omega_\v{k}t}+\hat{a}^\dagger_{\v{k},p}\cdot e^{i\v{k}\v{x}-i\omega_\v{k}t}) $
            \item Формула для электрического дипольного излучения: \\
            $ \d{I_p}{\Omega_\v{k}} = \frac{e^2 \omega_\v{k}^4}{2\pi c^3} \abs{\matrixel{\psi_2}{\hat{\v{x}}}{\psi_1}}^2;\; I = \sum_p \int d\Omega_\v{k}\d{I_p}{\Omega_\v{k}} = \frac{\hbar\omega}{\tau} $

        \end{itemize}
\item Уравнение Дирака  \begin{itemize}
            \item Уравнение Дирака: \\
            $ H_D = (\vec{\alpha}\v{p}c+\beta mc^2)+(-e\vec{\alpha}\v{A}+e\varphi) $\\
            $ \vec{\alpha} = \begin{pmatrix} 0 & \vec{\sigma} \\ \vec{\sigma} & 0 \end{pmatrix} =
            \left\{
                \begin{pmatrix}0&0&0&1\\0&0&1&0\\0&1&0&0\\1&0&0&0\end{pmatrix};
                \begin{pmatrix}0&0&0&\scalebox{0.75}[1.0]{-}i\\0&0&i&0\\0&\scalebox{0.75}[1.0]{-}i&0&0\\i&0&0&0\end{pmatrix};
                \begin{pmatrix}0&0&1&0\\0&0&0&\scalebox{0.75}[1.0]{-}1\\1&0&0&0\\0&\scalebox{0.75}[1.0]{-}1&0&0\end{pmatrix}
            \right\} $ \\
            $\beta = \begin{pmatrix} E & 0 \\ 0 & -E \end{pmatrix} =
            \begin{pmatrix}1&0&0&0\\0&1&0&0\\0&0&\scalebox{0.75}[1.0]{-}1&0\\0&0&0&\scalebox{0.75}[1.0]{-}1\end{pmatrix};\;
            \v{p} = -i\hbar\nabla $

        \end{itemize}
\end{enumerate}
\vfill\line(1,0){500}
\end{document}
