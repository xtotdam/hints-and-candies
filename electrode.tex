\documentclass[a4paper,12pt]{report}
\usepackage[T2A]{fontenc}
\usepackage[utf8]{inputenc}
\usepackage{amssymb,amsfonts}
\usepackage[fleqn]{amsmath}
\usepackage[russian,english]{babel}
\usepackage{anyfontsize}
\usepackage{graphicx}
\usepackage{fancyhdr}
\setlength{\headheight}{15.2pt}
\renewcommand{\headrulewidth}{0pt}
\pagestyle{fancy}

\usepackage{geometry}
\geometry{left=2cm}
\geometry{right=1.5cm}
\geometry{top=1cm}
\geometry{bottom=2cm}

\DeclareMathOperator{\sinc}{sinc}
\DeclareMathOperator{\Tr}{Tr}   %trace
\DeclareMathOperator{\Dim}{dim}   %trace
%\DeclareMathOperator{\tg}{tg}

\newcommand{\abs}[1]{\left| #1 \right|} % for absolute value
\newcommand{\norm}[1]{\lVert #1 \rVert} % for norm ||f||
\newcommand{\qnorm}[1]{\lVert #1 \rVert ^2} % for norm quadrat ||f||^2
\newcommand{\avg}[1]{\left< #1 \right>} % for average
\let\underdot=\d % rename builtin command \d{} to \underdot{}
\renewcommand{\d}[2]{\frac{d #1}{d #2}} % for derivatives
\newcommand{\dd}[2]{\frac{d^2 #1}{d #2^2}} % for double derivatives
\newcommand{\pd}[2]{\frac{\partial #1}{\partial #2}} % for partial derivatives
\newcommand{\pdd}[2]{\frac{\partial^2 #1}{\partial #2^2}} % for double partial derivatives
\newcommand{\pdc}[3]{\left( \frac{\partial #1}{\partial #2} \right)_{#3}} % for thermodynamic partial derivatives
\newcommand{\ket}[1]{\left| #1 \right>} % for Dirac bras
\newcommand{\bra}[1]{\left< #1 \right|} % for Dirac kets
\newcommand{\braket}[2]{\left< #1 \vphantom{#2} \right|\left. #2 \vphantom{#1} \right>} % for Dirac brackets
\newcommand{\matrixel}[3]{\left< #1 \vphantom{#2#3} \right| #2 \left| #3 \vphantom{#1#2} \right>} % for Dirac matrix elements
\newcommand{\comml}[2]{\left[ \hat{#1}, #2 \right]} % [A,...]
\newcommand{\commr}[2]{\left[ #1, \hat{#2} \right]} % [...,B]
\newcommand{\comm}[2]{\left[ \hat{#1}, \hat{#2} \right]} % for commutator for only two operators like [A,B]
\newcommand{\grad}[1]{\gv{\nabla} #1} % for gradient
\let\divsymb=\div % rename builtin command \div to \divsymb
\renewcommand{\div}[1]{\gv{\nabla} \cdot #1} % for divergence
\newcommand{\curl}[1]{\gv{\nabla} \times #1} % for curl

\newcommand{\praiseme}[1]{\scalebox{.4}{ \textcircled{\scalebox{.5}{CC}} \textcircled{\scalebox{.5}{BY}} \textcircled{\scalebox{.5}{SA}} #1, \the\day.\the\month.\the\year}}

\begin{document}
\rfoot{\praiseme{Xtotdam}}

\fontsize{20}{24}
\center{\bfseries Electrodynamics}
\line(1,0){500}
\fontsize{12}{15}
\begin{enumerate}[label=\textbf{\underline{\arabic*.}}]
\item $\left\{\begin{array}{l}
            \rot{H}=\frac{1}{c}\pd{\vec E}{t}+\frac{4\pi}{c}\vec{j}\\
            \rot{E}=-\frac{1}{c}\pd{\vec{H}}{t}\\
            \div{H}=0\\
            \div{E}=4\pi\rho
      \end{array}\right.$  ;
      $\vec{F}=e\vec{E}+\frac{e}{c}\vmult{V}{H} $
\inlineitem $\left\{\begin{array}{l}
            \oint\limits_L\scmult{H}{dl} = \frac{4\pi}{c} \int\limits_S \scmult{j}{dS}+\frac{1}{c}\pd{}{t}\int\limits_S\scmult{E}{dS}\\
            \oint\limits_L\scmult{E}{dl} = -\frac{1}{c}\pd{}{t}\int\limits_S\scmult{H}{dS}\\
            \oint\limits_S\scmult{H}{dS} = 0\\
            \oint\limits_S\scmult{E}{dS} = 4\pi\int\limits_V\rho dV
      \end{array}\right.$
\item $\left\{\begin{array}{l}
            \div{j}+\pd{\rho}{t}=0\\
            \equiv\;\partial_\alpha j^\alpha = 0\\
            \oint\limits_S\scmult{j}{dS}+\d{}{t}\int\limits_V\rho dV=0
      \end{array}\right.$  ;
      $\left\{\begin{array}{l}
            \frac{c}{4\pi}\operatorname{div}\vmult{E}{H}+\d{}{t}\frac{E^2+H^2}{8\pi}+\scmult{E}{j}=0\\
            \equiv\;\frac{c}{4\pi}\operatorname{div}\vec{\sigma}+\d{w}{t}+\scmult{E}{j}=0\\
            \frac{c}{4\pi}\oint\limits_S\vmult{E}{H}\vec{dS}+\d{}{t}E_f+\int\limits_V\scmult{E}{j}dV=0
      \end{array}\right.$
\item $\left\{\begin{array}{l}
            \vec{E}=-\gradn{\varphi}-\frac1{c}\pd{\vec{A}}{t}\\
            \vec{H}=\rot{A}
      \end{array}\right.$  ;
      $\left\{\begin{array}{l}
            \square\vec{A}=-\frac{4\pi}{c}\vec{j}(\vec{r},t)\\
            \square\varphi=-4\pi\rho(\vec{r},t)
      \end{array}\right.$  ;
      $\left\{\begin{array}{l}
            \vec{A}=\vec{A}'+\vec{\nabla}f(\vec{r},t)\\
            \varphi = \varphi'-\frac1{c}\pd{f(\vec{r},t)}{t}
      \end{array}\right.$  ;
      $ \partial_\alpha A^\alpha = 0 $
\item $\left\{\begin{array}{l}
            \text{ДСК: }\Delta = \pdd{}{x}+\pdd{}{y}+\pdd{}{z}\\
            \text{ЦСК: }\Delta = \frac1{r}\pd{}{r}(r\pd{}{r})+\frac1{r^2}\pdd{}{\varphi}+\pdd{}{z}\\
            \text{СфСК: }\Delta = \frac1{r^2}\pd{}{r}(r^2\pd{}{r})+\frac1{r^2\sin\theta}\pd{}{\theta}(\sin\theta\pd{}{\theta})+\frac1{r^2\sin^2\theta}\pdd{}{\varphi}
      \end{array}\right.$
\item $\left\{\begin{array}{l}
            \Delta\varphi(\vec{r})=-4\pi\rho\\
            \Delta\vec{A}(\vec{r})=-\frac{4\pi}{c}\vec{j}
      \end{array}\right.$  ;
      $\left\{\begin{array}{l}
            \varphi = \int\limits_V dV'\, \frac{\rho(\vec{r'},\tau)}{\abs{\vec{r}-\vec{r'}}}\\
            \vec{A} = \frac1{c}\int\limits_V dV'\, \frac{\vec{j}(\vec{r'},\tau)}{\abs{\vec{r}-\vec{r'}}}
      \end{array}\right.$  ;
      $ \tau = t-\frac{\abs{\vec{r}-\vec{r'}}}{c} $
\item $\left\{\begin{array}{l}
            \vec{d} = \int\limits_V\vec{r'}\rho(\vec{r'})dV' =\sum q_a \vec{r_a}\\
            \varphi(\vec{r}) = \scmult{r}{d}/r^3\\
            \vec{E} = (3\scmult{r}{d}\vec{r}-\vec{d}r^2)/r^5\\
            \varepsilon = -\scmult{E}{d}
      \end{array}\right.$  ;
      $\vec{d}$ независим от выбора СК, если $\sum q_a=0 $
\item $\left\{\begin{array}{l}
            \vec{m} = \frac1{2c}\int\limits_V dV'\,[\vec{r'}\times\vec{j}(\vec{r'})]\\
            \vec{A} = \vmult{m}{r}/r^3\\
            \vec{H} = (3\scmult{r}{m}\vec{r}-\vec{m}r^2)/r^5\\
            \varepsilon = -\scmult{H}{m}
      \end{array}\right.$
\item $\left\{\begin{array}{l}
            \rot{H}=\frac1{c}\pd{\vec{E}}{t}\\
            \rot{E}=-\frac1{c}\pd{\vec{H}}{t}\\
            \div{H}=0\\
            \div{E}=0\\
      \end{array}\right.$  ;
      $\left\{\begin{array}{l}
            \square\vec{E}=0\\
            \square\vec{H}=0\\
            \vec{E}=\vec{E_0}\exp(-i(\omega t-\scmult{k}{r}))\\
            \vec{H}=\vec{H_0}\exp(-i(\omega t-\scmult{k}{r}))
      \end{array}\right.$  ;
      $\left\{\begin{array}{l}
            \vmult{k}{H}=-\frac{\omega}{c}\vec{E}\\
            \vmult{k}{E}=\frac{\omega}{c}\vec{H}\\
            \scmult{k}{E}=\scmult{k}{H}=0
      \end{array}\right.$\\
      \begin{enumerate}
      \item $\vec{k}\perp\vec{E}\perp\vec{H} $
      \item $k=\frac{\omega}{c} $
      \item $\abs{\vec{E}} = \abs{\vec{H}}$
      \end{enumerate}
      Пусть $\vec{k}=k\vec{e_z} $\\
      \begin{enumerate}
      \item $\frac{E_y}{E_x}=\const\Rightarrow$ линейная поляризация
      \item $\frac{E_x^2}{E_0^2}+\frac{E_y^2}{E_0^2}=1 \Rightarrow$ круговая поляризация
      \item $\frac{E_x^2}{E_{x0}^2}+\frac{E_y^2}{E_{y0}^2}=1 \Rightarrow$ эллиптическая поляризация
      \end{enumerate}
\item $\left\{\begin{array}{l}
            \varphi(\vec{r}{\cdot}t)=(\vec{n},\dot{\vec{d}}(\tau))/cr\\
            \vec{A}=\dot{\vec{d}}(\tau)/cr
      \end{array}\right.$  ;
      $\left\{\begin{array}{l}
            \vec{E}=[[\ddot{\vec{d}}(\tau){\times}\vec{n}]{\times}\vec{n}]/c^2 r\\
            \vec{H}=[\ddot{\vec{d}}{\times}\vec{n}]/c^2 r
      \end{array}\right.$  ;
      $\tau=t-\frac{r}{c} $\\
      $I=\frac{2\ddot{\vec{d^2}}}{3c^3};\;\d{I}{\Omega}=\frac{[\ddot{\vec{d}}{\times}\vec{n}]^2}{4\pi c^3};\;
      r\ll \lambda \equiv \scmult{k}{r}\ll 1 $ - ближняя зона
\item $\vec{F}_{rad}=2q^2\dddot{\vec{r}}/3c^3 $  ;
      $\left\{\begin{array}{l}
            \abs{\vec{F}}\gg\abs{\vec{F}_{rad}}\Rightarrow\ddot{vec{r}}\approx\frac{\vec{F}}{m},\,
            \vec{F}_{rad}\approx\frac{2q^2}{3mc^3}\dot{\vec{F}}\\
            \abs{\dot{\vec{F}}}\approx\omega\abs{\vec{F}}\Rightarrow\frac{2q^2\omega}{3mc^3}\ll 1
      \end{array}\right.$ \\
      $\left\{\begin{array}{l}
            r_0=\frac{q^2}{mc^2}\\
            \lambda=\frac{2\pi c}{\omega}
      \end{array}\right.
      \Rightarrow\lambda\gg r_0-\text{дальняя (волновая) зона} $
\item $\left\{\begin{array}{l}
            x'=\gamma(x-Vt)\\
            y'=y\\z'=z\\
            t'=\gamma(t-\frac{Vx}{c})
      \end{array}\right.$
\inlineitem $\left\{\begin{array}{l}
            V_x'=(V_x-V)/(1-VV_x/c^2)\\
            V_y'=V_y\sqrt{1-\beta^2}/(1-VV_x/c^2)\\
            V_z'=V_z\sqrt{1-\beta^2}/(1-VV_x/c^2)
      \end{array}\right.$
\item $\vec{r'}=\begin{pmatrix}
            \gamma & {-}\beta\gamma & 0 & 0 \\
            {-}\beta\gamma & \gamma & 0 & 0 \\
            0 & 0 & 1 & 0 \\
            0 & 0 & 0 & 1
      \end{pmatrix}\vec{r}\;\;$  ;
      $\left\{\begin{array}{l}
            a'^0=\gamma(a^0-\beta a^1)\\
            a'^1=\gamma(a^1-\beta a^0)\\
            a'^2=a^2\\a'^3=a^3
      \end{array}\right.$  \\
      $\left\{\begin{array}{l}
            \vec{r^i}=\{ct,\vec{r}\}\\
            \vec{j^i}=\{c\rho,\vec{j} \}\\
            \vec{k^i}=\{\frac{\omega}{c},\vec{k} \}\\
            \vec{p^i}=\gamma m\{c,\vec{v} \}\\
            \vec{v^i}=\gamma\{c,\vec{v} \}\\
            \vec{w^i}=\d{\vec{v^i}}{\tau},\,d\tau=\sqrt{1-\beta^2}dt\\
            \vec{A^i}=\{\varphi,\vec{A} \}
      \end{array}\right.$  ;
      $\left\{\begin{array}{l}
            p_ip^i = \frac{\varepsilon^2}{c^2}-p^2=m^2c^2\\
            j_ij^i = c^2\rho^2-\vec{j}^2\\
            k_ik^i = \frac{\omega^2}{c^2}-\vec{k}^2\\
            u_iu^i = c^2\\
            r_ir^i = c^2t^2-\vec{r}^2
      \end{array}\right.$
\item $\left\{\begin{array}{l}
            \vec{E}'_\|=\vec{E}_\|\\
            \vec{H}'_\|=\vec{H}_\|\\
            \vec{E}'_\perp=\gamma(\vec{E}_\perp+\frac1{c}\vmult{V}{H})\\
            \vec{H}'_\perp=\gamma(\vec{H}_\perp-\frac1{c}\vmult{V}{E})
      \end{array}\right.$  ;
      $F_{ik}=\begin{pmatrix}
            0 & E_x & E_y & E_z \\
            -E_x & 0 & -H_z & H_y \\
            -E_y & H_z & 0 & -H_x \\
            -E_z & -H_y & H_x & 0
      \end{pmatrix}$  ;
      $\left\{\begin{array}{l}
            I_2=F_{ik}F^{ik}=2(E^2-H^2)\\
            I_4=2(E^2-H^2)+4\scmult{E}{H}^2
      \end{array}\right.$
\item $\left\{\begin{array}{l}
            \varepsilon=\gamma mc^2\\
            \vec{p}=\gamma m\vec{v}
      \end{array}\right.$  ;
      $\left\{\begin{array}{l}
            \vec{v}=\frac{\vec{p}c^2}{\varepsilon}=\frac{\vec{p}c}{\sqrt{m^2c^2+p^2}}\\
            \varepsilon^2=m^2c^4+p^2c^2\\
            mc^2=\sqrt{\varepsilon^2-p^2c^2}\\
            m=\gamma\frac{\varepsilon}{c^2}\\
            \vec{p}=\frac{\varepsilon\vec{v}}{c^2-v^2}
      \end{array}\right.$
\inlineitem $\mathscr{L}=-mc^2\sqrt{1-\beta^2}-e\varphi +\frac{e}{c}\scmult{V}{A} $
\item $\left\{\begin{array}{l}
            w = \frac1{8\pi}(E^2+H^2)-\text{плотность энергии}\\
            \vec{\sigma}=\frac{c}{4\pi}\vmult{E}{H}-\text{пл-ть потока энергии}\\
            \vec{g}=\frac1{4\pi c}\vmult{E}{H}=\frac{\vec{\sigma}}{c^2}-\text{пл-ть импульса}
      \end{array}\right.$
\item $\left\{\begin{array}{l}
            \d{\varepsilon}{t}=e\scmult{E}{V}\\
            \d{\vec{p}}{t}=e\vec{E}+\frac{e}{c}\vmult{V}{H}
      \end{array}\right.$  ;
      $\left\{\begin{array}{l}
            \d{}{t}\pd{\mathscr{L}}{\dot{q}}-\pd{\mathscr{L}}{q}=0\\
            \varepsilon=\sum p^iq^i - \mathscr{L}
      \end{array}\right.$  ;
      $mc\d{u^k}{\tau}=\frac{e}{c}F^{ki}u_i $
\end{enumerate}
\newpage
% \vfill\line(1,0){500}
\section*{Электродинамика сплошных сред}
\begin{enumerate}[label=\textbf{\underline{\arabic*.}}]
\item \textbf{Уравнения электростатики вещества}\\
      $\left\{\begin{array}{l}
            \div{D}=4\pi\rho(\vec{r})\\
            \rot{E}=0
      \end{array}\right.\;\;
      \vec{E}=-\vec{\nabla}\varphi;\;
      \Delta\varphi=-\frac{4\pi}{\varepsilon}\rho(\vec{r});\;
      \vec{D}=\varepsilon\vec{E} $\\
      Граничные условия: 2 диэлектрика\\
      $\left\{\begin{array}{l}
            E^I_\tau=E^{II}_\tau|_\Gamma\rightarrow
            \pd{\varphi_I}{\tau}=\pd{\varphi_{II}}{\tau}|_\Gamma\xrightarrow{\int}
            \varphi_I=\varphi_{II}|_\Gamma\\
            D^I_n=D^{II}_n=4\pi\rho_\text{пов. своб.}
      \end{array}\right.$\\
      $\vec{n}$ направлена из среды с большим номером в среду с меньшим.\\
      Граничные условия: диэлектрик + проводник\\
      $\left\{\begin{array}{l}
            E^I_\tau=E^{II}_\tau=0\\
            \rho_\text{инд. пов. своб.}=\frac1{4\pi}(D^I_n-D^{II}_n)=\frac{D^I_n}{4\pi}
      \end{array}\right. $\\
      $Q=\oint\limits_S \rho_\text{пов.}dS;\;
      dS_\text{ПСК}=R^2\int\limits_0^\pi\sin\theta d\theta\int\limits_0^{2\pi}d\varphi;\;
      \vec{F}_q=q\vec{E}_\text{возмущения в точке заряда q}$\\
      $\mathscr{E}_{int}=\mathscr{E}_\text{после}-\mathscr{E}_\text{до};\;
      d\mathscr{E}_{int}=-dA=-\int\limits_\infty^a F_z(z)dz\;\;$  \NB   $\;a\rightarrow z$
\item \textbf{Потенциалы и ёмкости}\\
      $c = \abs{\frac{q}{\varphi_1-\varphi_2}}$\\
      $\Delta\varphi = -4\pi\kappa\delta(y-b)\delta(x-a) \Rightarrow
      \varphi = -2\kappa \ln \sqrt{(x-a)^2+(y-b)^2}$\\
      Теорема взаимности: $\sum\limits_{a=1}^N q_a \varphi'_a = \sum\limits_{a=1}^N q'_a \varphi_a$
\item \textbf{Краевые задачи электростатики}\\
      $\Delta\varphi = 0$
      \begin{enumerate}
      \item СфСК\\
            $\left\{\begin{array}{l}
                  x = r\sin\theta\cos\varphi\\
                  y = r\sin\theta\sin\varphi\\
                  z = r\cos\theta
            \end{array}\right.\;\;$
            $\varphi = \left(\begin{array}{c}r^l\\\frac1{r^{l+1}}\end{array}\right)
            P^m_l(\cos\theta)\left\{\begin{array}{l}\sin m\varphi\\\cos m\varphi\end{array}\right.$\\
            $\begin{array}{c}\\\\P_n^m(\cos\theta)\\m=\overline{0,3}\\n=\overline{0,3}\end{array} =
            \begin{pmatrix}
                  1 & 0 & 0 & 0 & \cdots\\
                  \cos \theta & -\sin \theta & 0 & 0 & \cdots\\
                  \frac{1}{4} (3 \cos 2 \theta +1) & -3\sin \theta \cos \theta & 3 \sin ^2\theta & 0 & \cdots\\
                  \frac{1}{8} (3 \cos \theta+5 \cos 3 \theta ) &
                  -\frac{3}{2} \sin \theta \left(5 \cos ^2\theta-1\right) &
                  15 \sin ^2\theta \cos \theta &
                  -15 \sin ^3\theta & \cdots\\
                  \vdots & \vdots & \vdots & \vdots & \ddots
            \end{pmatrix} $
      \item ЦСК\\
            $\left\{\begin{array}{l}
                  x = r\cos\varphi\\
                  y = r\sin\varphi\\
                  z = z
            \end{array}\right. \;\;$
            $\varphi = \left(\begin{array}{c}r^n\\\frac1{r^n}\end{array}\right)
            \left(\begin{array}{l}\sin n\varphi\\\cos n\varphi\end{array}\right)\ln r$
      \end{enumerate}
      $\vec{f}_\text{ед. площ.}=\frac{\varepsilon E^2}{8 \pi}\vec{n};\;
      \vec{F}=\oint\limits_S\vec{f}dS$
\item \textbf{Электростатика диэлектриков}\\
      $\Delta\varphi = -\frac{4\pi}{\varepsilon}\rho $\\
      $\left\{\begin{array}{l}
            E^I_\tau=E^{II}_\tau|_\Gamma\rightarrow\varphi_I=\varphi_{II}|_\Gamma\\
            D^I_n=D^{II}_n=4\pi\rho_\text{пов. своб.}
      \end{array}\right. $\\
      $\vec{P}=\frac{\varepsilon -1}{4\pi}\vec{E};\;\;$ граница диэлектрика = плохое зеркало
\item \textbf{Силы, действующие на диэлектрик во внешнем поле}\\
      $\vec{F}_\text{ед. объёма}=\rho\vec{E}-\frac{E^2}{8\pi}\vec{\nabla}\varepsilon+
      \vec{\nabla}(\frac{E^2}{8\pi}\pd{\varepsilon}{\tau}\tau);\;\;
      \tau = \lim\limits_{\Delta V\rightarrow 0}\frac{\Delta m}{\Delta V}\text{ - плотность}$\\
      $\mathscr{F}_\text{тело из пров-ка}=\oint\limits_S \vec{f}_1 dS;\;
      \vec{f}_1=\frac{\varepsilon E^2}{8\pi}\vec{n}$\\
      $\mathscr{F}_\text{тело из диэл-ка}=\oint\limits_S \vec{f}_2 dS;\;
      \vec{f}_2=\frac{\varepsilon\scmult{E}{n}\vec{E}}{4\pi} -\frac{\varepsilon E^2}{8\pi}\vec{n}$
\item \textbf{Стационарные токи в проводниках}\\
      $\rot{H}=\frac{4\pi}{c}\vec{j}\Rightarrow
      \operatorname{div}\rot{H}=0=\frac{4\pi}{c}\div{j}\Rightarrow
      \div{j}=0 \Rightarrow$\\
      $\left\{\begin{array}{l}
            \div{B}=0 \Rightarrow \oint\limits_S \scmult{B}{dS}=0 \Rightarrow B^I_n=B^{II}_n\\
            \div{j}=0 \Rightarrow \oint\limits_S \scmult{j}{dS}=0 \Rightarrow j^I_n=j^{II}_n
      \end{array}\right. $\\
      $\vec{j}=\sigma(\vec{E}+\vec{E}_\text{стор.})$\\
      Принимаем $\vec{E}_\text{стор.} = 0 \Rightarrow \vec{E}=\frac{\vec{j}}{\sigma} \Rightarrow$
      $\left\{\begin{array}{l}
            E^I_\tau=E^{II}_\tau\\
            \frac{j^I_\tau}{\sigma_1}=\frac{j^{II}_\tau}{\sigma_2}
      \end{array}\right. $
\end{enumerate}
\end{document}
