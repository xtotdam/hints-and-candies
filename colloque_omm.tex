\documentclass[a4paper,12pt]{report}
\usepackage[T2A]{fontenc}
\usepackage[utf8]{inputenc}
\usepackage{amssymb,amsfonts}
\usepackage[fleqn]{amsmath}
\usepackage[russian,english]{babel}
\usepackage{anyfontsize}
\usepackage{graphicx}
\usepackage{fancyhdr}
\setlength{\headheight}{15.2pt}
\renewcommand{\headrulewidth}{0pt}
\pagestyle{fancy}

\usepackage{geometry}
\geometry{left=2cm}
\geometry{right=1.5cm}
\geometry{top=1cm}
\geometry{bottom=2cm}

\DeclareMathOperator{\sinc}{sinc}
\DeclareMathOperator{\Tr}{Tr}   %trace
\DeclareMathOperator{\Dim}{dim}   %trace
%\DeclareMathOperator{\tg}{tg}

\newcommand{\abs}[1]{\left| #1 \right|} % for absolute value
\newcommand{\norm}[1]{\lVert #1 \rVert} % for norm ||f||
\newcommand{\qnorm}[1]{\lVert #1 \rVert ^2} % for norm quadrat ||f||^2
\newcommand{\avg}[1]{\left< #1 \right>} % for average
\let\underdot=\d % rename builtin command \d{} to \underdot{}
\renewcommand{\d}[2]{\frac{d #1}{d #2}} % for derivatives
\newcommand{\dd}[2]{\frac{d^2 #1}{d #2^2}} % for double derivatives
\newcommand{\pd}[2]{\frac{\partial #1}{\partial #2}} % for partial derivatives
\newcommand{\pdd}[2]{\frac{\partial^2 #1}{\partial #2^2}} % for double partial derivatives
\newcommand{\pdc}[3]{\left( \frac{\partial #1}{\partial #2} \right)_{#3}} % for thermodynamic partial derivatives
\newcommand{\ket}[1]{\left| #1 \right>} % for Dirac bras
\newcommand{\bra}[1]{\left< #1 \right|} % for Dirac kets
\newcommand{\braket}[2]{\left< #1 \vphantom{#2} \right|\left. #2 \vphantom{#1} \right>} % for Dirac brackets
\newcommand{\matrixel}[3]{\left< #1 \vphantom{#2#3} \right| #2 \left| #3 \vphantom{#1#2} \right>} % for Dirac matrix elements
\newcommand{\comml}[2]{\left[ \hat{#1}, #2 \right]} % [A,...]
\newcommand{\commr}[2]{\left[ #1, \hat{#2} \right]} % [...,B]
\newcommand{\comm}[2]{\left[ \hat{#1}, \hat{#2} \right]} % for commutator for only two operators like [A,B]
\newcommand{\grad}[1]{\gv{\nabla} #1} % for gradient
\let\divsymb=\div % rename builtin command \div to \divsymb
\renewcommand{\div}[1]{\gv{\nabla} \cdot #1} % for divergence
\newcommand{\curl}[1]{\gv{\nabla} \times #1} % for curl

\newcommand{\praiseme}[1]{\scalebox{.4}{ \textcircled{\scalebox{.5}{CC}} \textcircled{\scalebox{.5}{BY}} \textcircled{\scalebox{.5}{SA}} #1, \the\day.\the\month.\the\year}}

\begin{document}
\rfoot{\praiseme{Xtotdam}}

\fontsize{20}{24}
\center{\bfseries OMM Colloque}
\line(1,0){500}
\fontsize{12}{15}
\begin{enumerate}
\item \textbf{Постановка задачи Гурса и ее решение}\\
      $\left\{\begin{array}{l}
            u_{xy}=f(x,y),\,x>0,\,y>0\\
            u(x,0)=\varphi(x),\,u(0,y)=\varphi(y)\\
            u(0,0)=\varphi_1(0)=\varphi_2(0)
      \end{array}\right. $\\$\Rightarrow
      u(x,y)=\varphi_1(x)+\varphi_2(y)-u(0,0)+ \int\limits_0^y \int\limits_0^x f(\xi,\eta)d\xi d\eta $\\
      Общая задача: \\
      $\left\{\begin{array}{l}
            u_{xy}+a(x,y)u_x+b(x,y)u_y+c(x,y)u=f(x,y),\,x>0,\,y>0\\
            u(x,0)=\varphi(x),\,u(0,y)=\varphi(y)\\
            u(0,0)=\varphi_1(0)=\varphi_2(0)
      \end{array}\right. $\\$\Rightarrow
      u(x,y)=\underbrace{\int\limits_0^y \int\limits_0^x F(\xi,\eta)d\xi d\eta}_{A\left[u\right]} + \Phi(x,y);
      \left\{\begin{array}{l}
            F = f-a(x,y)u_x-b(x,y)u_y-c(x,y)u\\
            \Phi = \varphi_1(x)+\varphi_2(y)-u(0,0)
      \end{array}\right. $\\$
      \Rightarrow u=A[u]+\Phi\;(u_n=A[u_{n-1}]+\Phi,\,u_0=0) $
\item \textbf{Постановка общей задачи Коши для гиперболического уравнения. Какими свойствами должна обладать кривая С, на которой ставятся дополнительные условия.}\\
      $\left\{\begin{array}{l}
            u_{xy}=f(x,y),\,(x,y)\in D^+ \;\;(1)\\
            u(x,y)=\varphi(x,y),\,(x,y)\in C\\
            \pd{u}{n}(x,y)=\psi(x,y),\,(x,y)\in C
      \end{array}\right. $\\
      $\Rightarrow u(M)=\frac{\varphi(A)+\varphi(B)}{2}-\frac{1}{2}\int\limits_{AB}(u_y dy-u_x dx)+\int\limits_D f(x,y)dxdy$\\
      \begin{enumerate}
      \item C - не характеристика уравнения (1)
      \item Любая характеристика (1) пересекает С только один раз
      \end{enumerate}
\item \textbf{Что произойдет, если характеристика уравнения общей задачи Коши пересечет кривую С, на которой заданы дополнительные условия, более чем в одной точке?}\\
      В таком случае потеряется произвольность $u(M_1)$, которая будет определяться как\\
      $ u(M_1)=\frac{u(A)V(A)+u(B_1)V(B_1)}{2}+\int\limits_{D_1}Vfdxdy-\frac{1}{2}\int\limits_{AB_1}Pdx+Qdy$\\
      $\left\{\begin{array}{l}
            V_{xy}=0,\,(x,y)\in D\\
            V_x|_{AM}=0,\,V_y|_{AM}=0,\,V(M)=1
      \end{array}\right.;\;
      \left\{\begin{array}{l}
            P[u,V]=V_x u-Vu_x\\
            Q[u,V]=V_y u-Vu_y
      \end{array}\right.$
\item \textbf{Определение и физический смысл функции Римана. Приведите примеры функции Римана}\\
      Определение - ?\\
      Физический смысл - $V(M,M_1)$ - функция влияния единичного импульса, сосредоточенного в $M_1$\\
      Примеры - ?
\item \textbf{Постановка задача Стефана. Примеры процессов, описываемых задачей Стефана}\\
      $\left\{\begin{array}{l}
            \pd{u_1}{t}=a_1^2\pdd{u_1}{x},\,0<x,\xi\\
            \pd{u_2}{t}=a_2^2\pdd{u_2}{x},\,\xi<x<\infty
      \end{array}\right.$\\
      $\left\{\begin{array}{l}
            u_1 = T_1,\,x=0\\
            u_2 = T,\,t=0
      \end{array}\right.$\\
      $u_1=u_2=0,\,x=\xi$\\
      $ k_1\pd{u_1}{x}|_{x=\xi}-k_2\pd{u_2}{x}|_{x=\xi}=\lambda\rho\d{\xi}{t}$
\item \textbf{В чем состоит метод подобия? Примеры задач, при решении которых целесообразно использовать метод подобия}\\
      ???
\item \textbf{Постановка задачи переноса вещества в двухфазной среде. Задача сорбции.}\\
      $\left\{\begin{array}{l}
            -V\pd{u}{x}=\pd{u}{t}+\pd{a}{t},\,x>0,\,t>0\\
            \pd{a}{t}=\beta(u-\gamma a),\,x>0,\,t>0\;\text{(ур-е кинетики сорбции)}\\
            a(x,0)=0,\,x\geq 0;\;u(x,0)=0,\,x>0\\
            u(0,t)=u_0,\,t\geq 0
      \end{array}\right.$\\
      $V$ - скорость газа;\\
      $a(x,t)$ - количество газа, поглощенного единицей объема сорбента;\\
      $u(x,t)$ - концентрация газа, находящегося в порах сорбента в слое $x$;\\
      $\pd{a}{t}$ - расход газа на увеличение сорбированного количества газа;\\
      $\pd{u}{t}$ - расход газа на повышение свободной концентрации в порах сорбента;\\
      $u_0$ - концентрация газа на входе;\\
      $\beta$ - кинетический коэффицент;\\
      $y$ - концентрация газа, находящегося в равновесии с сорбированным количеством газа.
\item \textbf{Переход к локальному времени в уравнениях переноса. Пример подобной замены переменной в задаче сорбции}\\
      $\left\{\begin{array}{l}
            t'=t-\frac{x}{V}\\
            x'=x
      \end{array}\right. \Rightarrow
      \left\{\begin{array}{l}
            \xi = \frac{x' \beta}{V}\\
            \tau = \beta t'
      \end{array}\right. \Rightarrow
      \left\{\begin{array}{l}
            \pd{u}{\xi}+\pd{a}{\tau}=0\\
            \pd{a}{\tau}=u-\gamma a\\
            a|_{\tau=-\xi}=0;\;u|_{\tau=-\xi}=0\\
            u(0,\tau)=u_0
      \end{array}\right.$
\item \textbf{Что такое изотерма сорбции? Приведите примеры}\\
      Изотерма сорбции - зависимость концентрации вещества в неподвижной фазе от его концентрации в подвижной при постоянной температуре $ \equiv a = f(y)|_{T=\const} $. Угол наклона изотермы сорбции определяет коэффициент распределения вещества между фазами. (Wiki)\\
      И. Ленгмюра $a=\frac{yu_0}{\gamma(u_0+py)} \xrightarrow[p\to 0]{}
      a=\underbrace{\frac1{\gamma}}_\text{к-та Генри}y$ - и. Генри
\item \textbf{Качественное различие решения задачи сорбции в линейном и нелинейном случае}\\
      ???
\item \textbf{Поведение на бесконечности решения уравнения Гельмгольца при различных знаках коэффициента С}\\
      $\Delta u+cu=-f$\\
      \begin{enumerate}
      \item $c = -\kappa^2 < 0$\\
            $ u^\pm(M) = \frac{1}{4\pi}\int\limits_D\frac{e^{\pm\kappa r_{QM}}}{r_{QM}}f(Q)dV_Q $,
            $f(M)$ - финитна, $ \operatorname{supp}f \subset D $\\
      \item $c=k^2,\,k=\bar{k}+i\bbar{k},\,\bbar{k}>0 $\\
            $\exists !$ решение, $\xrightarrow[\infty]{}0\,:\;
            u(M) = \frac{1}{4\pi}\int\limits_D\frac{e^{ikr_{QM}}}{r_{QM}}f(Q)dV_Q $ -
            при временной зависимости $e^{i\omega t}$ - это расходящаяся волна
      \item $c=k^2>0 $\\
            $u^\pm(M) = \frac{1}{4\pi}\int\limits_D\frac{e^{\pm kr_{QM}}}{r_{QM}}f(Q)dV_Q $ -
            оба одинаково убывают на $\infty$
      \end{enumerate}
\item \textbf{Сформулируйте для неограниченной области теорему единственности решения уравнения Гельмгольца в случае отрицательного коэффициента С}\\
      Классическое решение уравнения $\Delta u -\kappa^2 u = -f(M)$, равномерно стремящееся к нулю на бесконечности - единственно.
\item \textbf{Напишите условие излучения Зоммерфельда в трехмерном и двумерном случаях. Для чего ставятся условия излучения?}\\
      3D:
      $\left\{\begin{array}{l}
            u(M)=O(\frac{1}{r})\\
            \pd{u}{r}-iku=o(\frac{1}{r})
      \end{array}\right.\;\;\;$
      2D:
      $\left\{\begin{array}{l}
            u(M)=O(\frac{1}{\sqrt{r}})\\
            \lim\limits_{r\to\infty}(\pd{u}{r}-iku)=0
      \end{array}\right.$\\
      Условия ставятся для решений уравнения Гельмгольца
\item \textbf{Сформулируйте принцип предельного поглощения. Приведите пример постановки парциальных условий излучения}\\
      ???
\item \textbf{Как ставится задача математической теории дифракции?}\\
      $p_i=\const,\,\rho_i=\const,\;i=\overline{0,n}$\\
      $\left\{\begin{array}{l}
            \Delta V_i+k_i^2 V_i=-f_i,\,M\in D_i,\,i=\overline{0,n}\\
            V_i = V_0,\,M\in S_i\\
            p_i \pd{V_i}{n}=p_0\pd{V_0}{n},\,M\in S_i,\,i=\overline{1,n}\\
            V_0(M) = O(\frac{1}{r})\\
            \pd{V_0}{r}-ik_0 V_0 = o(\frac{1}{r})\\
            k_i^2=\frac{\rho_i}{p_i}\omega^2,\,i=\overline{0,n}
      \end{array}\right.$
\item \textbf{Рассмотрим задачу теплопроводности в бесконечной области. Если коэффициент теплопроводности зависит от температуры и обращается в ноль при нулевой температуре, то можно рассматривать финитные решения задачи. Если же уравнение линейно, то финитные решения рассматривать нельзя. Почему? Приведите пример задачи, в которой решение будет финитным}\\
      ???
\item \textbf{Приведите пример задачи с уравнением Буссинеска (описывающим уровень грунтовых вод над гидроупором), решение которой имеет автомодельный вид}\\
      ???
\item \textbf{Приведите примеры процессов, приводящих к модели Вольтера хищник - жертва. Как исследуется решение этой задачи на фазовой плоскости}\\
      ???
\item \textbf{Напишите линейное, линейное неоднородное и квазилинейное уравнение переноса. Составьте уравнения характеристик для этих случаев}\\
      ???
\item \textbf{Могут ли пересекаться характеристики в случае линейного и квазилинейного уравнения переноса? К какому качественному характеру решений и физическим результатам это приводит? В каких случаях необходимо строить обобщенное решение линейного и квазилинейного уравнения переноса?}\\
      ???
\item \textbf{Напишите условие на разрыве (условие Гюгонио) для квазилинейного уравнения переноса. В чем заключается метод характеристик решения квазилинейного уравнения переноса?}\\
      ???
\end{enumerate}
\vfill\line(1,0){500}
\end{document}
